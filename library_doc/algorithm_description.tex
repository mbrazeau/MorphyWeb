\documentclass[a4paper,12pt]{article}
\usepackage{hyperref}
\usepackage{enumitem}

\begin{document}

\noindent{\Large \bf Inapplicable algorithm description}


\section{First downpass}

% \begin{enumerate}
%     \item Enter the tree on a tip and get on its node.
%     \item Compare the character states between the left and the right descendant of the node.
%     \begin{enumerate}
%         \item \label{1stDp:case1:AND} If there is a \textbf{union} between both descendants set the node state as this union else go to \ref{1stDp:case1:OR}.
%         \begin{enumerate}
%             \item If there is an inapplicable token in the ancestral state \textit{and} that both descendants have also an inapplicable token, set the node state as the \textbf{intersection} of both descendants then go to \ref{1stDp:exitLoop}.
%         \end{enumerate}
%         \item \label{1stDp:case1:OR} Else set the node state as the \textbf{intersection} between both descendants.
%         \begin{enumerate}
%             \item If any of the descendants have an inapplicable token, add the inapplicable token to the node then go to \ref{1stDp:exitLoop}.
%         \end{enumerate}
%     \end{enumerate}
%     \item @@@ Activate so subtree states? % subtreeactives[i] |= (lft_active[i] | rt_active[i]) & MORPHY_IS_APPLICABLE;
%     \item \label{1stDp:exitLoop} Exit the first downpass.
% \end{enumerate}



%         ## Get the states in common between the descendants
%             ## If there is any states in common, set the node to be that one
%             ## If state in common is actually the inapplicable token, but that both descendants have an applicable, set it to be the union between the applicable states
%             ## Else set it to be the union of the descendants
%             ## If the node has inapplicable data but that both descendants have also applicable states, remove the inapplicable state from the node

\begin{enumerate}
    \item Enter on any of the tips of the tree and focus on its closest ancestor (hereafter referred as the ``focal node'') then go to \ref{1stDp:ANDdescen}.
    \item \label{1stDp:ANDdescen} \textbf{If} there is a state in common between both descendants of the focal node, \textbf{set} the focal node state to the states in common \textbf{then} go to \ref{1stDp:ANDinap}. \textbf{Else} go to \ref{1stDp:ORdescen}.
    \item \label{1stDp:ANDinap} \textbf{If} the state of the focal node is now only the inapplicable token but that both descendants also have an applicable token, \textbf{set} the focal node state to be the union of both descendants states without the inapplicable token; \textbf{then} go to \ref{1stDp:exitLoop}.
    \item \label{1stDp:ORdescen} \textbf{If} there is a no state in common between both descendants, \textbf{set} the focal node state to be the union of both descendants states; \textbf{then} go to \ref{1stDp:ORinap}.
    \item \label{1stDp:ORinap} \textbf{If} the focal node state has an inapplicable token but that both descendants have also applicable states, \textbf{remove} the inapplicable token from the the focal node state; \textbf{then} go to \ref{1stDp:exitLoop}.
    \item \label{1stDp:exitLoop} Go to the closest ancestor from the focal node and repeat from \ref{1stDp:ANDdescen} until all the nodes in the tree have been visited.
\end{enumerate}


\section{First uppass}

\begin{enumerate}
    \item Enter the tree on the root.
    \item @@@ Do some activation business.
    \item If the node has both an inapplicable token go to \ref{1stUp:nodeAppli} else go to \ref{1stUp:exitLoop}.
    \begin{enumerate}
        \item \label{1stUp:nodeAppli} If the node has an applicable token, go to \ref{1stUp:nodeAppli2} else go to \ref{1stUp:nodeInapp}.
        \begin{enumerate}
            \item \label{1stUp:nodeAppli2} If the ancestor has only the inapplicable token, set the node state to be only the inapplicable token; then go to \ref{1stUp:exitLoop}. Else go to \ref{1stUp:nodeAppli3}.
            \item \label{1stUp:nodeAppli3} Remove the inapplicable token from the node; then go to \ref{1stUp:exitLoop}.
        \end{enumerate}
        \item \label{1stUp:nodeInapp} The node is an inapplicable token; go to \ref{1stUp:nodeInapp1}.
        \begin{enumerate}
            \item \label{1stUp:nodeInapp1} If the ancestor has only the inapplicable token, set the node state to be only the inapplicable token; then go to \ref{1stUp:exitLoop}. Else go to \ref{1stUp:nodeInapp2}.
            \item \label{1stUp:nodeInapp2} If there is an \textbf{intersection} between the node's descendants states that is not inapplicable, set the node state to that \textbf{intersection}; then go to \ref{1stUp:exitLoop}. Else go to \ref{1stUp:nodeInapp3}.
            \item \label{1stUp:nodeInapp3} Set the node state to the inapplicable token only; then go to \ref{1stUp:exitLoop}.
        \end{enumerate}
    \end{enumerate}
    \item \label{1stUp:exitLoop} Exit the first uppass.
\end{enumerate}

\section{Second downpass}

\begin{enumerate}
    \item Enter the tree on a tip and get on its node.
    \item If the node contains at least one applicable state go to \ref{2ndDp:unionDescen} else, go to \ref{2ndDp:exitLoop}.
    \begin{enumerate}
        \item \label{2ndDp:unionDescen} If the \textbf{union} between both descendants have an applicable character, set the node state to this union without any inapplicable characters; then go to \ref{2ndDp:exitLoop}. Else go to \ref{2ndDp:interDescen}.
        \item \label{2ndDp:interDescen} Set the node state to be the \textbf{intersection} between the descendants and remove any eventual inapplicable tokens; then go to \ref{2ndDp:exitLoop}.
    \end{enumerate}
    \item @@@ Some activation business
    \item \label{2ndDp:exitLoop} Exit the second downpass.
\end{enumerate}

\section{Second uppass}


  
\begin{enumerate}
    \item If the node has no inapplicable token go to \ref{2ndUp:applicable} else go to \ref{2ndUp:inapplicableUnion}.
    \begin{enumerate}
        \item \label{2ndUp:applicable} If ancestral state also doesn't have an inapplicable token go to \ref{2ndUp:appli:case1}, else go to \ref{2ndUp:appli:case2}.
        \begin{enumerate}
            \item \label{2ndUp:appli:case1} If the \textbf{union} between the node and the ancestral state is equal to the ancestral state then set the node state to this union; then go to \ref{2ndUp:exitLoop}. Else go to \ref{2ndUp:exitLoop}. %and do some counting
            
            \item \label{2ndUp:appli:case2} If there is a \textbf{union} between the descendants then set the node to be the \textbf{intersection} between the \textbf{intersection} of the descendants and the \textbf{union} of the ancestral state (node = node | ((left | right) \& ancestral)); then go to \ref{2ndUp:exitLoop}. Else go to \ref{2ndUp:appli:case3}.
            
            \item \label{2ndUp:appli:case3} If there is an \textbf{intersection} between the descendants that has an inapplicable token then go to \ref{2ndUp:appli:case311}. Else go to \ref{2ndUp:appli:case32}.
            \begin{enumerate}
    
                    \item \label{2ndUp:appli:case311} If this intersection \textbf{unions} with the ancestral state then set the node state to be equal to the \textbf{intersection} between the descendants \textbf{unioned} with the ancestral state and \textbf{intersecting} with the ancestral state (((left | right) \& ancestral) | ancestral); then go to \ref{2ndUp:exitLoop}. Else go to \ref{2ndUp:appli:case312}.
                    
                    \item \label{2ndUp:appli:case312} Set the node state to be the \textbf{intersection} between the descendants \textbf{intersecting} with the ancestral character (left | right | ancestral) without any eventual inapplicable tokens; then go to \ref{2ndUp:exitLoop}. Else go to \ref{2ndUp:appli:case313}.
                    
                    \item \label{2ndUp:appli:case313} Set the node state to be the \textbf{intersection} between the node state and the ancestral state; then go to \ref{2ndUp:exitLoop}.

                    \item \label{2ndUp:appli:case32} If the \textbf{union} between node state and ancestral state equals the ancestral state set the node state to be this union; then go to \ref{2ndUp:exitLoop}. % and do some counting.

            \end{enumerate}
        \end{enumerate}
        \item \label{2ndUp:inapplicableUnion} If there is a \textbf{union} between the descendants, set the node state to be equal to this union; then go to \ref{2ndUp:exitLoop}. Else go to \ref{2ndUp:inapplicableNoUnion}. %and to some counting.
    \end{enumerate}
    \item \label{2ndUp:inapplicableNoUnion} do some counting
    \item \label{2ndUp:exitLoop} Exit second uppass.
\end{enumerate}




\end{document}